\documentclass[11pt]{report}
 
\usepackage{ragged2e}
\usepackage{amssymb}
\usepackage{amsmath}
\usepackage{graphicx}
\usepackage{bbm}
\usepackage{tablefootnote}
\usepackage{longtable}
\usepackage{csvsimple}
\usepackage{booktabs}

\graphicspath{ {hw1-data/} }

\begin{document}

\title{COMS 4721: Machine Learning for Data Science 
Home work 2}
\author{Chao Pang}
 
\maketitle

\section* {Problem 1}
Derive the solution for $\pi$ and each $\lambda_{y,d}$ by maximizing 

\[ \hat{\pi}, \hat{\lambda}_{0, 1:D}, \hat{\lambda}_{1, 1:D}  =  arg \underset{\hat{\pi}, \hat{\lambda}_{0, 1:D}, \hat{\lambda}_{1, 1:D} }{max} \sum_{i=1}^{n} \ln {p(y_{i}|\pi)} +  \sum_{d=1}^{D} \Biggl( \ln{p(\lambda_{0,d})} + \ln{p(\lambda_{1,d})}  + \sum_{i=1}^{n} \ln{p(x_{i,d} | \lambda_{y_{i},d}) \Biggr)} \]


\begin{align*} 
& \textit{Let} \;\; p(y | \pi)  = \pi^{y} (1 - \pi)^{1 - y}   \;\; \textit{where y $\in \{0, 1\}$}\\
& \textit{Let} \;\; p(\lambda | y)  = Gamma(2, 1) = \frac{ \lambda e^{-\lambda} } {\Gamma(2)}  \\
& \textit{Let} \;\; p(x | \lambda) = \frac{ \lambda^x} {x!}e^{-\lambda} \\
\end{align*}

\justify (a) Derive $\hat{\pi}$ using the objective above
\justify Solution: first plug in $p(y | \pi) $ and then take the derivative w.r.t \textbf{w}. 

\begin{align*} 
L &= \sum_{i=1}^{n} \Biggl( \ln{\pi^{ \mathbbm{1}\{ y_{i} = y \}  } (1 - \pi)^{ 1 - \mathbbm{1}\{ y_{i} = y \}  } } \Biggr) + \sum_{d=1}^{D} \Biggl( \ln{p(\lambda_{0,d})} + \ln{p(\lambda_{1,d})}  + \sum_{i=1}^{n} \ln{p(x_{i,d} | \lambda_{y_{i},d})} \Biggr) \\
L &= \sum_{i=1}^{n} \Biggl( \mathbbm{1}\{ y_{i} = y \}  \ln{\pi} + ({ 1 - \mathbbm{1}\{ y_{i} = y \}  }) \ln{(1 - \pi)}  \Biggr) \\
& + \sum_{d=1}^{D} \Biggl( \ln{p(\lambda_{0,d})} + \ln{p(\lambda_{1,d})}  + \sum_{i=1}^{n} \ln{p(x_{i,d} | \lambda_{y_{i},d})} \Biggr) \\
\\
\nabla{L_{\pi}} &= \sum_{i=1}^{n} \Biggl( \frac{ \mathbbm{1}\{ y_{i} = y \}  }{\pi} - \frac{ 1 - \mathbbm{1}\{ y_{i} = y \}  } { 1 - \pi}  \Biggr)  = 0 \\
\\
\pi &= \frac { \sum_{i=1}^{n} \mathbbm{1}\{ y_{i} = y \} } {n} \\ 
\end{align*}

\pagebreak

\justify (b) Derive $\hat{\lambda}_{y, d}$ using the objective above, leaving $y$ and $d$ arbitrary in your notation
\justify Solution: first plug in $p(\lambda | y)$ and $p(x | \lambda)$, then take the derivative w.r.t $\lambda_{y,d}$. 

\[
L= \sum_{i=1}^{n} \ln {p(y_{i}|\pi)} +  \sum_{d=1}^{D} \Biggl( \ln{ \frac{ \lambda_{0,d} \; e^{-\lambda_{0,d}} } {\Gamma(2)} } + \ln{ \frac{ \lambda_{1,d} \; e^{-\lambda_{1,d}} } {\Gamma(2)} }  + \sum_{i=1}^{n} \ln{\frac{ \lambda_{y_{i}, d}^{x_{i,d}}} {x_{i,d}!}e^ {-\lambda_{y_{i},d}} \Biggr)}
\]

\-
\justify Ignore the first term because $\sum_{i=1}^{n} \ln {p(y_{i}|\pi)}$ is not related $\lambda$. We can ignore the sum over the dimension $d$ and remove $d$ notation temporarily since all dimensions can be treated the same way, simplify and rewrite the equation as the following, 

\begin{small}
\[
\nabla{
\Biggl(
\ln{ \frac{ \lambda_{0} \; e^{-\lambda_{0}} } {\Gamma(2)} } + \ln{ \frac{ \lambda_{1} \; e^{-\lambda_{1}} } {\Gamma(2)} }  
+ \sum_{i=1}^{n} \ln{\frac{ \lambda_{0}^{x_{i}}} {x_{i}!}e^ {-\lambda_{0}}  \mathbbm{1}\{ y_{i} = 0 \} } 
+ \sum_{i=1}^{n} \ln{\frac{ \lambda_{1}^{x_{i}}} {x_{i}!}e^ {-\lambda_{1}}  \mathbbm{1}\{ y_{i} = 1} \}
\Biggr)
}
\]
\end{small}

\-
\justify Solving for $\lambda_{0, d}$
\[
\sum_{i=1}^{n} \frac{ x_{i} } {\lambda_{0}} \mathbbm{1}\{ y_{i} = 0 \}   -  \sum_{i=1}^{n} \mathbbm{1}\{ y_{i} = 0 \}  + \frac {1} {\lambda_{0}} - 1 = 0 
\]

\[
\lambda_{0} = \frac {\sum_{i=1}^{n} x_{i} \mathbbm{1}\{ y_{i} = 0 \} + 1} {\sum_{i=1}^{n} \mathbbm{1}\{ y_{i} = 0 \} + 1} 
\]

\- 
\justify $\lambda_{1,d}$ can be derived in the same way, so the the equation could be generalized using $y$. In addition, we put the dimension $d$ back to the equation, therefore we have the following,

\[
\lambda_{y, d} = \frac {\sum_{i=1}^{n} x_{i, d} \; \mathbbm{1}\{ y_{i} = y \} + 1} {\sum_{i=1}^{n} \mathbbm{1}\{ y_{i} = y \} + 1}  
\]
where $y \in \{0, 1\}$

\pagebreak

\section* {Problem 2}
\justify (a) the confusion matrix and the accuracy for naive Bayes classifier are shown below

\begin{center}
\begin{tabular}{ |c|c|c| } 
 \hline
  & Predicted 0  & Predicted 1 \\ 
 \hline
 Actual 0 & TN = 2295 & FP = 492 \\ 
  \hline
Actual 1  & FN = 99 &  TP = 1714 \\ 
 \hline
\end{tabular}
\end{center}

\justify where TN is true negative, FP is false positive, FN is false negative, and TP is true positive. The overall accuracy is \textbf{$87.15\%$}


\justify (b) \textbf{Figure~\ref{fig:lambda_stem}} shows that the $\lambda$ values in 16th dimension (\textbf{free}) and 52nd dimension (\textbf{!}) are much higher in spam than non-spam, indicating that the presence of these two words are highly suggestive of a spam email. 

\begin{figure}[h]
\includegraphics[scale=0.5]{hw2-data/Bayes_classifier/lambda_stem.png}
\centering
\caption{Plot stem plot for 54 dimensions for spam and non-spam}
\label{fig:lambda_stem}
\end{figure}

\pagebreak

\justify (c) plot the training function per each iteration for 10 runs in the same figure for logistic regression
\begin{figure}[h]
\includegraphics[scale=0.4]{hw2-data/Bayes_classifier/logistic_regression_learning_objective.png}
\centering
\caption{plot the training function each iteration for 10 runs for logistic regression}
\label{fig:logistic_regression_learning_objective}
\end{figure}


\justify (d) Derive the update for $w_{t+1}$ for Newton's method

\[
L(w) \approx L^{'} (w)   \equiv L(w_{t}) + (w - w_{t})^{T} \nabla{L(w_{t})} + \frac{1}{2} (w - w_{t})^{T} \nabla^{2}{L(w_{t})} (w - w_{t})
\]

\justify Solution: set $w_{t+1} = arg\;max_{w} L^{'} (w) $ and $\nabla^{2}{L(w_{t})}$ is symmetric

\begin{align*}
L^{'} (w) =& L(w_{t}) + w^{T} \nabla{L(w_{t})}  - w_{t}^{T} \nabla{L(w_{t})} \\
&+ \frac{1}{2} \Biggl( w^{T} \nabla^{2}{L(w_{t})} w  - w^{T}\nabla^{2}{L(w_{t})}  w_{t} - w^{T}_{t} \nabla^{2}{L(w_{t})} w + w^{T}_{t} \nabla^{2}{L(w_{t})}  w_{t}\Biggr) \\
=& L(w_{t}) + w^{T} \nabla{L(w_{t})}  - w_{t}^{T} \nabla{L(w_{t})} \\
&+ \frac{1}{2} \Biggl( w^{T} \nabla^{2}{L(w_{t})} w  - 2w^{T}\nabla^{2}{L(w_{t})}  w_{t} + w^{T}_{t} \nabla^{2}{L(w_{t})}  w_{t} \Biggr)
\end{align*}

\justify Solve for w
\[
\nabla{L^{'}(w)} =  \nabla{L(w_{t}})^{T} + w^{T} \nabla^{2}{L(w_{t})} - w_{t}^{T} \nabla^{2}{L(w_{t})} = 0
\]

\[
\nabla^{2}{L(w_{t})}  w =   \nabla^{2}{L(w_{t})} w_{t} - \nabla{L(w_{t}})  \\
\]

\[
w=  w_{t} - \nabla^{2}{L(w_{t})}^{-1} \; \nabla{L(w_{t}}) 
\]

\justify we have the first order derivative w.r.t to w 

\[
\nabla{L(w}) = \sum_{i=1}^{n} (1 - \sigma_{i}(y_{i} \cdot w)) y_{i} x_{i}
\]

\justify this can be written in the matrix form
\[
\nabla{L(w}) = X^{T}SY
\]
where $S$ is a n $\times$ n diagonal matrix, where $S_{ii} = 1 - \sigma_{i} ({y_{i} \cdot w})$

\-
\justify We need to compute the second order derivative w.r.t to w, the Hessian matrix

\[
\nabla{L(w}) = \sum_{i=1}^{n} (1 - \frac{e^{y_{i} x_{i}^{T} w}} {(1 + e^{y_{i} x_{i}^{T} w})} ) y_{i} x_{i}
\]

\[
\nabla{L(w}) = \sum_{i=1}^{n} ( 1 - \frac{1} {(1 + e^{-y_{i} x_{i}^{T} w})} ) y_{i} x_{i}
\]

\[
\nabla^{2}{L(w}) = \sum_{i=1}^{n} - \frac{ e^{-y_{i} x_{i}^{T} w } } {(1 + e^{-y_{i} x_{i}^{T} w})^{2}}  y_{i}^{2} x_{i} x_{i}^{T}
\]

\[
\nabla^{2}{L(w}) = \sum_{i=1}^{n} - \sigma_{i} ({y_{i} \cdot w})( 1 - \sigma_{i} ({y_{i} \cdot w})) x_{i} x_{i}^{T}
\]
\-
\justify $\nabla^{2}{L(w})$ can be written in the matrix quadratic form 
\[
\nabla^{2}{L(w}) = - X^{T} M X
\]
where $M$ is a n $\times$ n diagonal matrix, where $M_{ii} = \sigma_{i} ({y_{i} \cdot w})( 1 - \sigma_{i} ({y_{i} \cdot w}))$

\-
\justify Finally, we can plug the gradient and Hessian matrix back in

\begin{align*}
w &=  w_{t} + (X^{T} M X)^{-1} (X^{T}SY)\\
&=  (X^{T} M X)^{-1} X^{T} (M X w_{t} + SY)\\
\end{align*}

\-
\justify \textbf{Figure~\ref{fig:newton_method_learning_objective}} for plotting training function per each iteration for 10 runs for Newton's method is shown in the next page

\begin{figure}[h]
\includegraphics[scale=0.40]{hw2-data/Bayes_classifier/newton_method_learning_objective.png}
\centering
\caption{plot the training function per each iteration for 10 runs for Newton's method}
\label{fig:newton_method_learning_objective}
\end{figure}

\pagebreak

\justify (e) the confusion matrix and the accuracy for Newton's method are shown below

\begin{center}
\begin{tabular}{ |c|c|c| } 
 \hline
  & Predicted 0  & Predicted 1 \\ 
 \hline
 Actual 0 & TN = 2645 & FP = 142 \\ 
  \hline
Actual 1  & FN = 263 &  TP = 1550 \\ 
 \hline
\end{tabular}
\end{center}
where TN is true negative, FP is false positive, FN is false negative, and TP is true positive. The overall accuracy is \textbf{$91.20\%$}

\pagebreak

\section* {Problem 3}

\justify (a) the table below shows the rmse associated with every combination of \textbf{b} and $\sigma^{2}$. 

\begin{center}
\begin{tabular}{lrrrrrrrrrr}
\toprule
\;\;\;  $\sigma^2$ &   0.1 &   0.2 &   0.3 &   0.4 &   0.5 &   0.6 &   0.7 &   0.8 &   0.9 &   1.0 \\
b  &       &       &       &       &       &       &       &       &       &       \\
\midrule
5  &  1.97 &  1.93 &  1.92 &  1.92 &  1.92 &  1.93 &  1.93 &  1.94 &  1.95 &  1.95 \\
7  &  1.92 &  1.90 &  1.91 &  1.92 &  1.92 &  1.93 &  1.94 &  1.95 &  1.96 &  1.97 \\
9  &  1.90 &  1.90 &  1.92 &  1.93 &  1.95 &  1.96 &  1.97 &  1.98 &  1.98 &  1.99 \\
11 &  1.89 &  1.91 &  1.94 &  1.96 &  1.97 &  1.99 &  2.00 &  2.01 &  2.01 &  2.02 \\
13 &  1.90 &  1.94 &  1.96 &  1.99 &  2.00 &  2.01 &  2.02 &  2.03 &  2.04 &  2.05 \\
15 &  1.91 &  1.96 &  1.99 &  2.01 &  2.03 &  2.04 &  2.05 &  2.06 &  2.07 &  2.07 \\
\bottomrule
\label{table:rmse_table}
\end{tabular}
\end{center}

\justify (b) Based on the cross validation results, the combination that generated the lowest RMSE in the test set is when $b=11$ and $\sigma^2=0.1$. Below is the metrics generated from the first homework using polynomial regression
\begin{itemize}
	\item 1st order polynomial regression: $\lambda = 0, RMSE_{min} \approx 2.634$
	\item 2nd order polynomial regression: $\lambda = 51, RMSE_{min} \approx 2.13$
	\item 3rd order polynomial regression: $\lambda = 52, RMSE_{min} \approx 2.10$
\end{itemize} 

\justify It seems that Gaussian Process had a better performance in terms of RMSE than the polynomial regression in the test set. 


\pagebreak
\justify (c) Plot the predicted mean and the original data points from the training data
\begin{figure}[h]
\includegraphics[scale=0.40]{hw2-data/Gaussian_process/car_weight_rmse.png}
\centering
\caption{Predicted mean and the original data points}
\label{fig:car_weight_rmse}
\end{figure}

\end{document}